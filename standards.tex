\documentclass[a4paper,11pt,twoside]{article}
\usepackage[italian]{babel}
\usepackage[utf8]{inputenc}
\usepackage[T1]{fontenc}

%\usepackage{bookman}
%\usepackage{libertine}
%\usepackage{quattrocento}
\usepackage{avant}
\renewcommand{\familydefault}{\sfdefault}
\usepackage[paper=a4paper,top=0.8cm,bottom=1.2cm,right=2cm,left=2cm]{geometry} % margini
\usepackage{graphicx}
\usepackage{titlesec}
\usepackage{framed,wrapfig}
\usepackage{parskip}
\usepackage{enumitem}

\setlist[itemize]{leftmargin=*}
\setlist{itemsep=.5em}
\newcommand{\titoletto}[1]{\begin{centering}{\small #1\par}\end{centering}\vspace{1em}}
\renewcommand\arraystretch{1.25}

\begin{document}
\pagestyle{empty}
\begin{minipage}{.32\textwidth}
	\begin{framed}
		\titoletto{Catorci lentissimi}
		\begin{tabular}{l p{3cm}}
			CPU & Pentium 4\\
			RAM & 1 GiB\\
			HDD & 80 GB \newline (o 40+40 GB)\\
		\end{tabular}
	\end{framed}
\end{minipage}%
\hfill
\begin{minipage}{.32\textwidth}
	\begin{framed}
		\titoletto{Calcolatori decenti}
		\begin{tabular}{l p{3cm}}
			CPU & Core 2 Duo, i3\\
			RAM & 2 GiB\\
			HDD & 160 GB o pi\`u\newline \\
		\end{tabular}
	\end{framed}
\end{minipage}%
\hfill
\begin{minipage}{.32\textwidth}
	\begin{framed}
		\titoletto{Calcolatori buoni}
		\begin{tabular}{l p{3cm}}
			CPU & Core 2 Quad, i5\\
			RAM & 4 GiB\\
			HDD & Quanto basta\newline \\
		\end{tabular}
	\end{framed}
\end{minipage}%
\vspace{.016\textwidth}
\begin{minipage}{.32\textwidth}
	\begin{framed}
		\titoletto{Server, workstation}
		\begin{tabular}{l p{3cm}}
			CPU & Xeon, Opteron\\
			RAM & Quella che c'\`e\\
			HDD & RAID di qualche genere o SSD\newline \\
		\end{tabular}
	\end{framed}
\end{minipage}

\section{Installazione}
\begin{enumerate}
	\item Avviare Xubuntu > F4 > ``Installazione OEM per configuratori hardware robe cose'' > Installa
	\item Nome che chiede nella prima schermata: \texttt{WEEE Open}
	\item Partizionamento automatico
	\item Password: \texttt{open}
	\item A fine installazione riavviare
	\item \texttt{sudo timedatectl set-ntp true}
	\item \texttt{sudo apt update}
	\item \texttt{sudo apt upgrade}
	\item \texttt{sudo apt install vlc}
	\item Spegnere \emph{senza} fare ``Prepare for shipping''
\end{enumerate}

\section{Dove scrivere i codici}

\subsection{Case di fissi}
\begin{itemize}
	\item Su etichetta WEEE Open, con pennarello nero. Attaccare l'etichetta davanti. Se impossibile, sopra. Se ancora impossibile, in quel posto.
	\item Dietro, su una parte del case (no PSU, no IO shield), pennarello qualsiasi
	\item Sul lato interno del fianco apribile (anche dell'altro se tende a cascare da solo), pennarello qualsiasi
	\item Sulla gabbia di HDD/ODD, in vista col case aperto, se possibile, pennarello qualsiasi
\end{itemize}

\subsection{Portatili}
Su etichetta WEEE Open attaccata sotto una parte non rimuovibile (no sportelli) o sul retro dello schermo, con pennarello nero

\subsection{RAM}
Con pennarello bianco, 1 carattere per ogni chip tanto sono uguali. Oppure con pennarello qualsiasi su un'area vuota della PCB, se c'\`e. Oppure attacare etichetta bianca tagliata.

\subsection{Schede PCI (grafiche, audio, etc...)}

Sul bracket in una zona libera, sia all'interno che all'esterno, in modo che sia visibile e dritto a pc montato, se possibile.

Da qualche altra parte sulla PCB, va bene anche etichetta bianca tagliata.

\subsection{Lettori CD (ODD)}

Sopra, sul retro (con pennarello bianco se \`e nero), sul fianco verso il retro in modo che da montato sia leggibile

\subsection{Altri componenti}

Seguire il buon senso, basta che si legga quando sono montati...

\section{Pulizia del calcolatore}

\begin{itemize}
	\item Rimuovere etichette brutte e inutili (scritte a mano, strappate, etc...), soprattutto se ostacolano le nostre. Non rimuovere assolutamente CIB e seriali Windows!
	\item Togliere la polvere usando i propri polmoni
	\item Cambiare pasta termica se necessario, scrivere nelle note se fatto cos\`i non si rif\`a
\end{itemize}

\end{document}

